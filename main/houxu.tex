
{\kaishu
\begin{center}
\LARGE 美人赋
\end{center}
\par
{\centering 问世间,情是何物,直教人死生相许。

}
\par
晚来闲适,步余马于洛水之滨。时来晚风清,江水渌,芳草纤纤,百花团簇。渺远景微波鳞鳞,及近尔花鸟相容。思散梦始,俯仰察观。睹远水涟漪处生一仙子,宓妃之貌,皇英之姿。髣髴兮若轻云之蔽月,飘飖兮若流风之回雪。修虞姿态,蹁跹飞仙。动之明波,思之魄飞。江流萦渚,汀兰为之弥香。罗衣璀璨,百花与之争艳。蘅皋芳蔼,气若幽兰。游龙戏水,翔凤高鸣。婀娜柔情,灵妃顾笑。凌波微步,罗袜生尘。无奈天长水远,求之无迹,欲教传媒,青鸟衔恨,思若流波,怛兮在心。
\par
“春草碧色,春水渌波,送君南浦,伤之如何。秋露如珠,秋月如珪,明月白露,光阴往来,与子之别,思心徘徊。”
\par
谁言思之不伤兮,谅天地之杳冥。不得寄此银笺兮,泣尽中夜而未曦。晚秋风景倍凄凉,万里愁生高楼上。衰草凝烟乍不绿,枯木笼月影徒伤。玉楼楼高人独倚,鸳鸯瓦冷柳色空。秋气凄目桂枝落,寒江流水伴愁浓。不知玉人在何处,今昔何夕,情寄何方?宿雁偏落,寒芦深处。鸳鸯只影,冰字寒生。落叶逐流水,总寄相思去,却又是红衣狼藉风雨处。回首埋愁地,泪随流水急,又更兼三分葬花天气。香奩畔,翠翘在,人何还。梦觉晓来倚虚幌,烛光残尽,镜中空白九分头。噫,秋风秋月秋煞人,可怜落红飘尽,风吹冷香轻。几声寒鸦归去,而今触绪添愁,点滴芭蕉心。偷窥天上明月,小立人间风寒,共谁同圆缺。最是不胜一处相思一处闲愁。
\par
{\centering 泠泠岬岸水,流不尽古今情,唯有一首红豆寄相思。

}
\par
醉死不相见,恨杀红线老!海水直下万里深,谁人不言相思苦。美女如花隔云端,上有青明之高天,下有渌水之波澜,天长地远魄飞苦,梦魄不到关山难。长相思,泪始干,长相思,摧心肝。朝发洧盘兮秣驷于扶桑之下,八龙为余先驱兮翼祥云而踌躇。倚阊阖而还望兮寄忧思于南浦,送美人兮伤一别,不念此去千里。若乃苍梧山崩,湘水阻绝,九嶷倾颓,湘竹泣血。问此间,花为谁开,竹为谁青,泪为谁洒,梦为谁踟蹰。于今直念古者海枯石烂之忠贞;至如天路高寒兮汉祚衰,山河崩颓兮计难归,雁南飞兮独漂泊,弦音绝兮无人会。肠断兮思愔愔,胡笳十八拍,拍不散眉弯;或乃三国兵甲,乱世烽烟,生怜玉骨,钗没黄沙,一计赴国,两虎相争,含恨梦碎,极目天涯。焚蝶续貂蝉,万世沧桑,写不尽,红颜泪;若夫沈园一见,词赋相合,十年生死,斑鬓削磨,贯是东风吹雨恶,不道人间情转薄,天若有情今安在,偏叫人间沧桑多;偿有相如醉剑,文君鸣琴,五弦绿绮,音节高清,交颈颉颃,一低一昂;又如铜雀台畔漳水流,古来欢情几度秋,漓漓浮华随逝水,彤云北度又登楼,云聚散,梦悠悠,金龙玉凤绕空楼,千古功名今安在,惟有旧时明月,旧时圆缺;又还道,成郢烟土,细腰争舞。金屋匿娇,长门空赋。人生初见,憔悴有时。长恨歌舞,比翼连枝。圆圆曲罢,冲冠一怒。后庭花开,今翻摇落,“花开花落不长久,落红满地归寂中”。任娇红繁彩,妖绿染世,只看取菡萏,那花中最洁贞。

}
% \bibliography{ANN}%bib文件名称
% \end{document}
