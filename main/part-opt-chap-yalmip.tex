% % \part{优化模型}
% % \chapter{优化工具Yalmip简介}

\chapter{优化工具Yalmip简介}
\section{一些Yalmip可解的优化模型}
    \par
    下面,我们来介绍一个MATLAB优化工具箱(重量级) - Yalmip。先来列举一些Yalmip可以求解的优化问题:
    \par
    (1)线性规划LP
    \begin{align*}
    &\min_x \ c^\mathrm{T}x\\
    &s.t.\left\{
    \begin{aligned}
    &Ax = b\\
    &x_i \geqslant 0
    \end{aligned}
    \right.
    \end{align*}
    MATLAB使用单纯形法、内点法和动态序列算法来求解线性规划问题,求解命令为x=linprog(c,A,b)。
    \par
    (2)二次规划QP
    \begin{align*}
    &\min_x \ x^\mathrm{T} Hx + x^\mathrm{T}p\\
    &s.t.\quad Ax \geqslant b
    \end{align*}
    MATLAB使用内点凸包算法、信赖域反射算法和动态序列算法求解二次规划,求解命令为x=quadprog(\\H,c,A,b).
    \par
    (3)二阶锥规划SOCP
    \begin{align*}
    &\min_x \ c^\mathrm{T}x\\
    &s.t.\left\{
    \begin{aligned}
    &Ax = b\\
    &A_ix - b_i \preceq L^{m_i}
    \end{aligned}
    \right.
    \end{align*}
    其中:$L^{m_i}$是$R^{m_i}$中的二阶锥。
    \par
    (4)线性半定规划SDP
    \begin{align*}
    &\min_x \ c \cdot x\\
    &s.t.\left\{
    \begin{aligned}
    &A_ix = b_i\\
    &x \succeq 0
    \end{aligned}
    \right.
    \end{align*}
    \par
    (5)非线性半定规划
    \begin{align*}
    &\min_x \ c^\mathrm{T}x\\
    &s.t. \quad F_0 + \sum_{i = 1}^mx_iF_i \succeq 0,\quad i = 1,2,\dots,m
    \end{align*}
    其中:$b_i$为实数;$c,A_i,F_i$为$n\times n$实对称矩阵。
    \par
    (6)行列式规划
   \begin{align*}
    &\min_x \ c ^\mathrm{T} x + \log \det G(x)^{-1}\\
    &s.t.\left\{
    \begin{aligned}
    &G(x) \geqslant 0\\
    &F(x) \geqslant 0
    \end{aligned}
    \right.
    \end{align*}
    其中:$x\in R^m$,$G:R^m \rightarrow R^{l\times l}$,$F:R^m\rightarrow R^{n\times n}$,$G_i = G_i^\mathrm{T}$,$F_i = F_i^\mathrm{T}$,$i = 0,1,\dots,m$
    \begin{align*}
    G(x) = G_0 + x_1G_1 + \dots,+x_mG_m\\
    H(x) = H_0 + x_1H_1 + \dots,+x_mH_m
    \end{align*}
    一个专用于行列式规划的工具箱是“maxdet”,可参考\cite{Maxdetguide}。
    \par
    (7)几何规划
    \begin{align*}
    &\min_x \ f_0(x)\\
    &s.t.\left\{
    \begin{aligned}
    &f_i(x) \leqslant 1,\quad i = 1,\dots,m\\
    &h_i(x) = 1,\quad i = 1,\dots,p
    \end{aligned}
    \right.
    \end{align*}
    其中:$f_i(x)$是正项函数,$x\in R^n$。
    % 由于优化部分我们没有介绍几何规划,这里我们给出几何规划的一个示例
    % \begin{align*}
    % &\min_{x,y,z} \ x^{-1}y^{-\frac{1}{2}}z^{-1} + 2.3xz+4xyz   \\
    % &s.t.\left\{
    % \begin{aligned}
    % &\frac{1}{3} x^{-2}y^{-2} + \frac{4}{3}y^{\frac{1}{2}}z^{-1} \leqslant 1 \\
    % &x+2y+3z \leqslant 1\\
    % & \frac{1}{z}xy = 1
    % \end{aligned}
    % \right.
    % \end{align*}
    一个专用于几何规划的工具箱是“GGPLAB”,可参考\cite{GGPLAB}。
    \par
    Yalmip还可以求解整数规划、混合整数规划和全局规划等等最优化问题,这里不再详细介绍。下面,我们来看Yalmip的用法。
\section{Yalmip应用实例}
    \subsection{Yalmip的安装}
        \begin{lstlisting}[language = Matlab]
        %% Installation YALMIP
        folderName = fullfile(matlabroot,'toolbox','yalmip');%[pwd filesep 'yalmip']
        mkdir(folderName)
        cd(folderName)
        urlwrite('https://github.com/yalmip/yalmip/archive/master.zip','yalmip.zip');
        unzip('yalmip.zip','yalmip');
        p = genpath(folderName);
        addpath(p);
        savepath
        \end{lstlisting}
    \subsection{解线性规划和二次规划}
        \par
        用Yalmip求解线性规划和二次规划。以线性回归问题为例,我们设定线性回归模型为
        \begin{align*}
        &y = xw+\varepsilon\\
        &\hat{y} = xw
        \end{align*}
        其中:$x$是$n\times p$矩阵。现在要求使下面的目标最小的$w$
        \begin{enumerate}
        \item $\min\limits_w \ J_1(w) = ||y - \hat{y}||^\infty = \max_i \{|y_i - \hat{y}_i|\}$
        \item $\min\limits_w \ J_2(w) = ||y - \hat{y}||^1 = \sum_{i = 1}^n |y_i - \hat{y}_i|$
        \item $\min\limits_w \ J_3(w) = ||y - \hat{y}||^2 =  \sqrt{\sum_{i = 1}^n (y_i - \hat{y}_i)^2} = (y - \hat{y})^\mathrm{T}(y - \hat{y})$
        \end{enumerate}
        更一般的,我们可以设置目标为离差$y - \hat{y}$的$p$阶范数
        \begin{align*}
        \min_w \ ||y - \hat{y}||^p
        \end{align*}
        \par
        对上面的目标1、2和3,目标2为线性规划问题,目标3为二次规划问题,并且这些优化问题不存在约束条件。上述3个目标的Yalmip的程序为
        \begin{lstlisting}[language = Matlab]
        %% 线性规划&二次规划----线性回归示例
        w = [1 2 3 4 5 6]';
        n = (0:0.02:2*pi)';
        x = [sin(n) sin(2*n) sin(3*n) sin(4*n) sin(5*n) sin(6*n)];
        e = (-4+8*rand(length(n),1));
        e(100:115) = 30;
        y = x*w+e;
        plot(n,y);
        %第一种方法
        what = sdpvar(6,1);
        residuals = y-x*what;
        optimize([],norm(residuals,inf));
        w_Linf = value(what);
        optimize([],norm(residuals,1));
        w_L1 = value(what);
        optimize([],norm(residuals,2));
        w_L2 = value(what);
        plot(n,[y x*w_L1 x*w_L2 x*w_Linf]);
        %第二种方法
        bound = sdpvar(length(residuals),1);
        Constraints = [-bound <= residuals <= bound];
        optimize(Constraints,sum(bound));
        w_L1 = value(what);
        %
        optimize([],residuals'*residuals);
        w_L2 = value(what);
        %
        bound = sdpvar(1,1);
        Constraints  = [-bound <= residuals <= bound];
        optimize(Constraints,bound);
        w_Linf = value(what);
        %绘图
        plot(n,[y x*w_L1 x*w_L2 x*w_Linf]);
        \end{lstlisting}
    \subsection{解二阶锥规划}
        \par
        我们仍然以回归问题为例,考虑一般形式的岭回归
        \begin{align*}
        \min_w \ ||y - xw||^2 + \lambda||w||^2
        \end{align*}
        为了方便,这里我们取$\lambda = 1$,于是优化问题变为
        \begin{align*}
        \min_w \ ||y - xw||^2 + ||w||^2
        \end{align*}
        将上面的优化问题变为二阶锥SOCP问题,有
        \begin{align*}
        &\min_w  \ u+v\\
        &s.t.\left\{
        \begin{aligned}
        &||y-xw||^2 \leqslant u\\
        &||w||^2 \leqslant v
        \end{aligned}
        \right.
        \end{align*}
        上面问题的程序为
        \begin{lstlisting}[language = Matlab]
        %% 二阶锥规划--线性回归示例
        sdpvar u v
        %方法1
        F = [cone(y-x*what,u), cone(what,v)];
        solvesdp(F,u + v);
        %方法2
        F = [norm(y-x*what,2) <= u, norm(what,2) <= v];
        solvesdp(F,u + v);
        %方法3
        solvesdp([],norm(y-x*what,2) + norm(what,2));
        \end{lstlisting}
    \subsection{解半定规划}
        \par
        考虑如下半定规划问题
        \begin{align*}
        &\min_w \ c^\mathrm{T}x\\
        &s.t.\left\{
        \begin{aligned}
        &A_1x = b_1\\
        &A_2x \geqslant b_2\\
        &F_0 + x_1F_1+\dots+x_pF_p \geqslant 0
        \end{aligned}
        \right.
        \end{align*}
        其中:$F_0+x_1F_1 + \dots+x_pF_p\geqslant 0$是一个LMI(Linear Matrix Inequality)。我们将模型中的量设置为
        \begin{align*}
        &F_0 =
        \begin{bmatrix}
        2& -0.5 & -0.6\\
        -0.5& 2 & 0.4 \\
        -0.6 & 0.4 & 3
        \end{bmatrix}
        ,\quad
        F_1 =
        \begin{bmatrix}
        0 & 1 & 0\\
        1 & 0 & 0\\
        0 & 0 & 0
        \end{bmatrix}
        \\
        &F_2 =
        \begin{bmatrix}
        0 & 0 & 1\\
        0 & 0 & 0\\
        1 & 0 & 0
        \end{bmatrix}
        ,\quad
        F_3 =
        \begin{bmatrix}
        0 & 0 & 0\\
        0 & 0 & 1\\
        0 & 1 & 0
        \end{bmatrix}
        \end{align*}
        以及
        \begin{align*}
        & 0.7 \leqslant x_1 \leqslant 1\\
        & 0 \leqslant x_2 \leqslant 0.3\\
        & 0 \leqslant x_3\\
        & x_1+x_2+x_3 = 1
        \end{align*}
        \par
        令$t=c^\mathrm{T}x$,则有
        \begin{align*}
        & \min_x \ t\\
        & s.t.
        \left\{
        \begin{aligned}
        & A_1x = b_1\\
        & A_2x \geqslant b_2\\
        & tI - (x_1F_1+x_2F_2+x_3F_3) \geqslant 0
        \end{aligned}
        \right.
        \end{align*}
        其中:$x = (x_1,x_2,x_3,t)^\mathrm{T}$,$A=[1,1,1,0]^\mathrm{T}$,$b=1$,$b_2 = (0.1,-1,0,-0.3,0)^\mathrm{T}$,
        \begin{math}
        A_2=
        \left(
        \begin{smallmatrix}
        1& 0 & 0 & 0\\
        -1& 0 & 0 & 0\\
        0& 1 & 0 & 0\\
        0 &-1 & 0 & 0\\
        0& 0 & 1 &0
        \end{smallmatrix}
        \right)
        \end{math}
        。利用Yalmip求解上述NLSDP(非线性半定规划)的程序为
        \begin{lstlisting}[language = Matlab]
        %% 半定规划
        t = sdpvar(1);
        x = sdpvar(3,1,'full');
        F0 = [2,-0.5,-0.6;-0.5,2,0.4;-0.6,0.4,3];
        F1 = [0 1 0; 1 0 0; 0 0 0];
        F2 = [0 0 1;0 0 0;1 0 0];
        F3 = [0 0 0;0 0 1;0 1 0];
        a = [sum(x) == 1];%等式约束
        b = [0.7 <= x(1) <= 1, 0 <= x(2)<=0.3, x(3)>=0];%不等式约束
        c = [t*eye(3) - (F0 + x(1)F1 + x(2)F(2)+ x(3)F3)>= 0 ]%LMI约束
        obj = t;
        constraint = [a,b,c];
        solvesdp(constraint,obj);
        xbest = double(x);
        \end{lstlisting}
    \subsection{解行列式规划}
        \par
        给定两个椭圆
        \begin{align*}
        E_1 = \{x|x^\mathrm{T}P_1x \leqslant 1\}\\
        E_2 = \{x|x^\mathrm{T}P_2x \leqslant 1\}
        \end{align*}
        找到包含这两个$E_1,E_2$的最小椭圆$E = x^\mathrm{T}x \leqslant 1$。由于椭圆的体积与$\det p$成正比,所以我们的目标是
        \begin{align*}
        & \max_P \ \det P\\
        & s.t.\left\{
        \begin{aligned}
        & P \leqslant t_1P_1\\
        & P \leqslant t_2P_2\\
        & 0 \leqslant t_1 \leqslant 1\\
        & 0 \leqslant t_2 \leqslant 1
        \end{aligned}
        \right.
        \end{align*}
        目标函数$\min_P \ -\det P$是非凸的,但是单调转换可以使其变为凸问题,例如:$\min_P\ -\log \det P$。Yalmip采用如下变化
        \begin{align*}
        \min_P\ -\det (P)^{\frac{1}{m}}
        \end{align*}
        其中:$m$为$P$的维数,这里$m = 2$。其求解程序为
        \begin{lstlisting}[language = Matlab]
        %% 行列式规划
        n = 2;
        P1 = randn(2);P1 = P1*P1';
        P2 = randn(2);P2 = P2*P2';
        t = sdpvar(2,1);
        P = sdpvar(n,n);
        F = [1 >= t >=0, t(1)*P1 - P >= 0, t(2)*P2 >= 0];
        obj = -geomean(P);
        sol = optimize(F,obj);
        x = sdpvar(2,1);
        plot(x'*value(P1)*x,[]),hold on
        plot(x'*value(P2)*x,[])
        plot(x'*value(P)*x,[])
        % 或者
        optimize(F - logdet(P),sdpsetting('solver','sdpt3'));
        \end{lstlisting}
    \subsection{解一般凸规划}
        \par
        考虑如下一般的凸规划问题
        \begin{align*}
        & \max_x \ -\sum \log(b - Ax)\\
        & s.t.\quad Ax \leqslant b
        \end{align*}
        令$y = \log(b-Ax)$,则
        \begin{align*}
        & \min_x \ -\sum y\\
        & s.t. \quad e^y <= b-Ax
        \end{align*}
        上述问题的Yalmip程序为
        \begin{lstlisting}[language = Matlab]
        %% 一般凸规划
        n = 5;m = 20;
        A = randn(m,n);
        b = rand(m,1)*m;
        x = sdpvar(n,1);
        y = sdpvar(m,1);
        constraint = [exp(y <= b-Ax)];
        obj = -sum(y);
        sol = optimize(constraint,obj);
        \end{lstlisting}
    \subsection{整数规划}
        \par
        考虑如下整数规划问题
        \begin{align*}
        & \min_x \ c^\mathrm{T}x\\
        & s.t.\left\{
        \begin{aligned}
        & Ax \leqslant b\\
        & Aeq x \leqslant beq\\
        & x \geqslant x_m\\
        & x\text{为整数}
        \end{aligned}
        \right.
        \end{align*}
        \par
        MATLAB旧版本用bintprog命令来求解0-1规划,新版本MATLAB用intprog命令来求0-1规划、整数规划和混合整数规划。
        上述问题的Yalmip求解程序为
        \begin{lstlisting}[language = Matlab]
        %% 整数规划
        c = -[2 1 4 3 1]';
        A = [0 2 1 4 2;3 4 5 -1 -1];
        b = [54; 62];
        Aeq = [];beq = [];
        xm = [0, 0 ,3.52, 0.678, 2.57]';
        x = sdpvar(5,1);
        obj = c'*x;
        constraint = [Ax <= b, xm <= x, integer(x)];
        sol = optimize(constraint ,obj);
        \end{lstlisting}
        \par
        另外,我们考虑在线性回归中,待求参数$w$是整数的情况
        \begin{align*}
        &\min_w \ ||y - wx||^2\\
        &s.t.\quad w\text{为整数}
        \end{align*}
        \begin{lstlisting}[language = Matlab]
        what = intvar(6,1);%what = sdpvar(6,1);
        residuals = y-x*what;
        optimize([],norm(residuals,2));
        w_L2 = value(what);
        plot(n,[y x*w_L1 x*w_L2 x*w_Linf]);
        \end{lstlisting}
    \subsection{非凸二次规划}
        \par
        考虑如下非凸二次规划
        \begin{align*}
        & \min_x\ x^\mathrm{T}Qx \\
        & s.t.\quad -1 \leqslant x \leqslant 1
        \end{align*}
        其求解程序为
        \begin{lstlisting}[language = Matlab]
        %% 非凸二次规划
        Q = magic(5);
        x = sdpvar(5,1);
        ops1 = sdpsettings('solver','bmibnb','bmibnb.maxiter',1000,...
        'bmibnb.uppersolver','fmincon');
        ops2 = sdpsettings('solver','moment','moment.order',3);
        ops3 = sdpsettings('solver','moment','moment.order',2);
        ops4 = sdpsettings('solver','kktqp');
        for n = 1:10
          Q = magic(n);
          x = sdpvar(n,x);
          sol = optimize([-1<=x<=1],x'*Q*x,ops1);
          comptimes(n,1) = sol.solvertine;
          sol = optimize([-1<=x<=1],x'*Q*x,ops2);
          comptimes(n,2) = sol.solvertine;
          sol = optimize([-1<=x<=1,x.*x<=1],x'*Q*x,ops3);
          comptimes(n,3) = sol.solvertine;
          sol = optimize([-1<=x<=1],x'*Q*x,ops4);
          comptimes(n,4) = sol.solvertine;
        end
        semilogy(1:n,comptimes)
        \end{lstlisting}

